% Options for packages loaded elsewhere
\PassOptionsToPackage{unicode}{hyperref}
\PassOptionsToPackage{hyphens}{url}
\PassOptionsToPackage{dvipsnames,svgnames,x11names}{xcolor}
%
\documentclass[
  letterpaper,
  DIV=11,
  numbers=noendperiod]{scrartcl}

\usepackage{amsmath,amssymb}
\usepackage{iftex}
\ifPDFTeX
  \usepackage[T1]{fontenc}
  \usepackage[utf8]{inputenc}
  \usepackage{textcomp} % provide euro and other symbols
\else % if luatex or xetex
  \usepackage{unicode-math}
  \defaultfontfeatures{Scale=MatchLowercase}
  \defaultfontfeatures[\rmfamily]{Ligatures=TeX,Scale=1}
\fi
\usepackage{lmodern}
\ifPDFTeX\else  
    % xetex/luatex font selection
\fi
% Use upquote if available, for straight quotes in verbatim environments
\IfFileExists{upquote.sty}{\usepackage{upquote}}{}
\IfFileExists{microtype.sty}{% use microtype if available
  \usepackage[]{microtype}
  \UseMicrotypeSet[protrusion]{basicmath} % disable protrusion for tt fonts
}{}
\makeatletter
\@ifundefined{KOMAClassName}{% if non-KOMA class
  \IfFileExists{parskip.sty}{%
    \usepackage{parskip}
  }{% else
    \setlength{\parindent}{0pt}
    \setlength{\parskip}{6pt plus 2pt minus 1pt}}
}{% if KOMA class
  \KOMAoptions{parskip=half}}
\makeatother
\usepackage{xcolor}
\setlength{\emergencystretch}{3em} % prevent overfull lines
\setcounter{secnumdepth}{-\maxdimen} % remove section numbering
% Make \paragraph and \subparagraph free-standing
\makeatletter
\ifx\paragraph\undefined\else
  \let\oldparagraph\paragraph
  \renewcommand{\paragraph}{
    \@ifstar
      \xxxParagraphStar
      \xxxParagraphNoStar
  }
  \newcommand{\xxxParagraphStar}[1]{\oldparagraph*{#1}\mbox{}}
  \newcommand{\xxxParagraphNoStar}[1]{\oldparagraph{#1}\mbox{}}
\fi
\ifx\subparagraph\undefined\else
  \let\oldsubparagraph\subparagraph
  \renewcommand{\subparagraph}{
    \@ifstar
      \xxxSubParagraphStar
      \xxxSubParagraphNoStar
  }
  \newcommand{\xxxSubParagraphStar}[1]{\oldsubparagraph*{#1}\mbox{}}
  \newcommand{\xxxSubParagraphNoStar}[1]{\oldsubparagraph{#1}\mbox{}}
\fi
\makeatother


\providecommand{\tightlist}{%
  \setlength{\itemsep}{0pt}\setlength{\parskip}{0pt}}\usepackage{longtable,booktabs,array}
\usepackage{calc} % for calculating minipage widths
% Correct order of tables after \paragraph or \subparagraph
\usepackage{etoolbox}
\makeatletter
\patchcmd\longtable{\par}{\if@noskipsec\mbox{}\fi\par}{}{}
\makeatother
% Allow footnotes in longtable head/foot
\IfFileExists{footnotehyper.sty}{\usepackage{footnotehyper}}{\usepackage{footnote}}
\makesavenoteenv{longtable}
\usepackage{graphicx}
\makeatletter
\def\maxwidth{\ifdim\Gin@nat@width>\linewidth\linewidth\else\Gin@nat@width\fi}
\def\maxheight{\ifdim\Gin@nat@height>\textheight\textheight\else\Gin@nat@height\fi}
\makeatother
% Scale images if necessary, so that they will not overflow the page
% margins by default, and it is still possible to overwrite the defaults
% using explicit options in \includegraphics[width, height, ...]{}
\setkeys{Gin}{width=\maxwidth,height=\maxheight,keepaspectratio}
% Set default figure placement to htbp
\makeatletter
\def\fps@figure{htbp}
\makeatother

% =============================
% Preamble for Rguroo Book Project
% =============================

% --- Required packages ---
\usepackage{hyperref}
\usepackage[most]{tcolorbox}
\usepackage{xcolor}

% --- Colors (Okabe–Ito palette) ---
\definecolor{OIblue}{HTML}{0072B2}
\definecolor{OIsky}{HTML}{56B4E9}
\definecolor{OIgreen}{HTML}{009E73}
\definecolor{OIorange}{HTML}{E69F00}
\definecolor{OIverm}{HTML}{D55E00}
\definecolor{OIpurple}{HTML}{CC79A7}
\definecolor{OIblack}{HTML}{000000}

% --- Inline macros (for .var, .def, .fun, etc.) ---
\newcommand{\Var}[1]{\textcolor{OIblue}{\textit{#1}}}
\newcommand{\Def}[1]{\textcolor{OIpurple}{\textit{#1}}}
\newcommand{\Des}[1]{\textcolor{OIorange}{#1}}
\newcommand{\Data}[1]{\textcolor{OIsky}{\textit{#1}}}
\newcommand{\Dpd}[1]{\textcolor{OIgreen}{\textit{#1}}}
\newcommand{\Fun}[1]{\textcolor{OIverm}{\textit{#1}}}
\newcommand{\Dialog}[1]{\textcolor{OIblue}{\textbf{#1}}}
\newcommand{\Repo}[1]{\textcolor{OIgreen}{\textbf{\textit{#1}}}}
\newcommand{\Ans}[1]{\textcolor{OIblack}{\textbf{#1}}}
\newcommand{\Typein}[1]{%
  \colorbox{blue!2}{\texttt{\textcolor{black}{#1}}}%
}
\newtcbox{\Button}{on line,
  colback=gray!5,
  coltext=black,
  colframe=gray!40,
  boxrule=0.4pt,
  left=2pt,right=2pt,top=1pt,bottom=1pt,
  arc=1mm, boxsep=0pt, nobeforeafter}

% --- QCallout boxes for PDF (LaTeX rendering) ---
\newtcolorbox{qcalloutbox}[2][]{%
  enhanced, breakable,
  colback=blue!2, colframe=blue!40!black,
  title={#2}, fonttitle=\bfseries,
  left=6pt, right=6pt, top=6pt, bottom=6pt,
  boxsep=4pt, arc=2mm, #1
}

\NewDocumentEnvironment{QCallout}{ m m }{%
  \hypertarget{#1}{}%
  \begin{qcalloutbox}{#2}%
}{%
  \end{qcalloutbox}
}
\KOMAoption{captions}{tableheading}
\makeatletter
\@ifpackageloaded{caption}{}{\usepackage{caption}}
\AtBeginDocument{%
\ifdefined\contentsname
  \renewcommand*\contentsname{Table of contents}
\else
  \newcommand\contentsname{Table of contents}
\fi
\ifdefined\listfigurename
  \renewcommand*\listfigurename{List of Figures}
\else
  \newcommand\listfigurename{List of Figures}
\fi
\ifdefined\listtablename
  \renewcommand*\listtablename{List of Tables}
\else
  \newcommand\listtablename{List of Tables}
\fi
\ifdefined\figurename
  \renewcommand*\figurename{Figure}
\else
  \newcommand\figurename{Figure}
\fi
\ifdefined\tablename
  \renewcommand*\tablename{Table}
\else
  \newcommand\tablename{Table}
\fi
}
\@ifpackageloaded{float}{}{\usepackage{float}}
\floatstyle{ruled}
\@ifundefined{c@chapter}{\newfloat{codelisting}{h}{lop}}{\newfloat{codelisting}{h}{lop}[chapter]}
\floatname{codelisting}{Listing}
\newcommand*\listoflistings{\listof{codelisting}{List of Listings}}
\makeatother
\makeatletter
\makeatother
\makeatletter
\@ifpackageloaded{caption}{}{\usepackage{caption}}
\@ifpackageloaded{subcaption}{}{\usepackage{subcaption}}
\makeatother

\ifLuaTeX
  \usepackage{selnolig}  % disable illegal ligatures
\fi
\usepackage{bookmark}

\IfFileExists{xurl.sty}{\usepackage{xurl}}{} % add URL line breaks if available
\urlstyle{same} % disable monospaced font for URLs
\hypersetup{
  pdftitle={Home},
  colorlinks=true,
  linkcolor={blue},
  filecolor={Maroon},
  citecolor={Blue},
  urlcolor={Blue},
  pdfcreator={LaTeX via pandoc}}


\title{Home}
\author{}
\date{}

\begin{document}
\maketitle


\subsection*{Preface}\label{preface}
\addcontentsline{toc}{subsection}{Preface}

This is test of my macros: Inline examples (should be colored): - The
variable is \Var{x}. - The defined term is \Def{standard deviation}. -
The descriptor is \Des{Frequency}. - The dataset is \Data{Cars}. - The
dropdown selection is \Dpd{Mean}. - The function is \Fun{lm}. - The
dialog name is \Dialog{Descriptive Statistics}. - The repository is
\Repo{Rguroo Datasets}. - The answer is \Ans{42}.

I hope you find this book useful in teaching statistics. When writing
this book, I tried to follow the
\href{Retrieved\%20from\%20http://www.amstat.org/education/gaise/GAISECollege_\%20Recommendations.pdf}{GAISE
Standards (GAISE recommendations}.

\begin{itemize}
\item
  Teach statistical thinking.
\item
  Focus on conceptual understanding.
\item
  Integrate real data with a context and a purpose.
\item
  Foster active learning.
\item
  Use technology to explore concepts and analyze data.
\item
  Use assessments to improve and evaluate student learning
\end{itemize}

To this end, I ask students to interpret the results of their
calculations. I incorporated the use of technology (R Studio) for most
calculations. Because of that you will not find me using any of the
computational formulas for standard deviations or correlation and
regression since I prefer students understand the concept of these
quantities. Also, because I utilize technology you will not find the
standard normal table, Student's t-table, binomial table, chi-square
distribution table, and F-distribution table in the book. Another
difference between this book and other statistics books is the order of
hypothesis testing and confidence intervals. Most books present
confidence intervals first and then hypothesis tests. I find that
presenting hypothesis testing first and then confidence intervals is
more understandable for students. Lastly, I have de-emphasized the use
of the z-test. In fact, I only use it to introduce hypothesis testing,
and never utilize it again. Two samples should be emphasized over one
sample test. Lastly, to aid student understanding and interest, most of
the homework and examples utilize real data with multiple variables. The
beauty of multiple variables, is that you can ask the students to
investigate different analysis with different variables. This way
students can work with data and come up with connections of asking
questions and using data to answer the questions. Again, I hope you find
this book useful for your introductory statistics class.

\subsection*{Mathematical Knowledge
Assumed}\label{mathematical-knowledge-assumed}
\addcontentsline{toc}{subsection}{Mathematical Knowledge Assumed}

I want to make a comment about the mathematical knowledge that I assumed
the students possess. The course for which I wrote this book has a
higher prerequisite than most introductory statistics books. However, I
do feel that students can read and understand this book as long as they
can read critically. I do not show how to create most of the graphs, but
all graphs are created with R Studio. So I hope the mathematical level
is appropriate for your course.

\subsection*{Technology Used}\label{technology-used}
\addcontentsline{toc}{subsection}{Technology Used}

The technology that I utilized for creating the graphs and statistical
analysis is R Studio. This is a statistical software that are used by
statisticians and so using it gives students skills they may need in the
future. Please feel free to use any other technology that is more
appropriate for your students. Do make sure that you use some
technology. I worked on the \href{https://www.statprep.org/}{StatPREP
project} and there are Little Apps that can be used to explore data.
There are also activities that can be used in your classes that utilize
the Little Apps on the website.

\subsection*{Acknowledgments}\label{acknowledgments}
\addcontentsline{toc}{subsection}{Acknowledgments}

I would like to thank the following people for taking their valuable
time to review the book. Their comments and insights improved this book
immensely.

\begin{itemize}
\item
  Daniel Kaplan, Macalester College
\item
  Jane Tanner, Onondaga Community College
\item
  Rob Farinelli, College of Southern Maryland
\item
  Carrie Kinnison, retired engineer
\item
  Sean Simpson, Westchester Community College
\item
  Kim Sonier, Coconino Community College
\item
  Jim Ham, Delta College
\item
  Brian Birgen, Wartburg College
\item
  Christopher Cunningham, Elgin Community College
\item
  Kendra Feinstein, Tacoma Community College
\item
  David Straayer, Tacoma Community College
\item
  Students of Coconino Community College
\item
  Students of Elgin Community College
\item
  Students of Tacoma Community College
\item
  Students of Wartburg College
\end{itemize}

I also want to thank Coconino Community College for granting me a
sabbatical so that I would have the time to write the book. On a
personal note, I wanted to thank my brother, John Matic, his wife
Jenelle, and their children Hannah and Eli for their hospitality when
writing the first edition. In addition to allowing my family access to
their home, John provided numerous examples and data sets for business
applications in this book. I inadvertently left this thank you out of
the first edition of the book, His help and his family's hospitality
were invaluable to me. Lastly, I want to thank my husband Rich and my
son Dylan for supporting me in this project. Without their love and
support, I would not have been able to complete the book.

\subsection*{New to the Fourth Edition}\label{new-to-the-fourth-edition}
\addcontentsline{toc}{subsection}{New to the Fourth Edition}

The additions to this edition mostly involve format changes and other
edits to make the textbook more accessible for students with visual
disabilities. Have a textbook that is accessible to all is very
important to me, so please let me know if more changes need to be made.
Minor changes and corrections were also made. One change is that every
hypothesis test and confidence interval has assumptions that must be
true to make the inference valid. Instead of calling them assumptions
though, I decided to call them conditions to remove confusion about
other assumptions.

\subsection*{Packages Needed for r
studio}\label{packages-needed-for-r-studio}
\addcontentsline{toc}{subsection}{Packages Needed for r studio}

You will need the following packages installed and loaded in r Studio:
arm, HNANES, MASS, mosaic, Weighted.Desc.Stat.

\subsection*{License}\label{license}
\addcontentsline{toc}{subsection}{License}

Creative Commons Attribution Sharealike.

2025 Kathryn Kozak

\includegraphics{Creative_commons.png}

ISBN:




\end{document}
